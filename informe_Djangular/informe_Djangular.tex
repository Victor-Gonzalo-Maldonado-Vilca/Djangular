%package list
\documentclass{article}
\usepackage[top=3cm, bottom=3cm, outer=3cm, inner=3cm]{geometry}
\usepackage{multicol}
\usepackage{graphicx}
\usepackage{url}
%\usepackage{cite}
\usepackage{hyperref}
\usepackage{array}
%\usepackage{multicol}
\newcolumntype{x}[1]{>{\centering\arraybackslash\hspace{0pt}}p{#1}}
\usepackage{natbib}
\usepackage{pdfpages}
\usepackage{multirow}    
\usepackage[normalem]{ulem}
\useunder{\uline}{\ul}{}
\usepackage{svg}
\usepackage{xcolor}
\usepackage{listings}
\lstdefinestyle{ascii-tree}{
    literate={├}{|}1 {─}{--}1 {└}{+}1 
  }

\lstset{basicstyle=\ttfamily,
  showstringspaces=false,
  commentstyle=\color{red},
  keywordstyle=\color{blue}
}
%\usepackage{booktabs}
\usepackage{caption}
\usepackage{subcaption}
\usepackage{float}
\usepackage{array}

\usepackage{enumitem}


\newcolumntype{M}[1]{>{\centering\arraybackslash}m{#1}}
\newcolumntype{N}{@{}m{0pt}@{}}


%%%%%%%%%%%%%%%%%%%%%%%%%%%%%%%%%%%%%%%%%%%%%%%%%%%%%%%%%%%%%%%%%%%%%%%%%%%%
%%%%%%%%%%%%%%%%%%%%%%%%%%%%%%%%%%%%%%%%%%%%%%%%%%%%%%%%%%%%%%%%%%%%%%%%%%%%
\newcommand{\itemEmail}{vmaldonadov@unsa.edu.pe}
\newcommand{\itemStudent}{Victor Gonzalo Maldonado Vilca}
\newcommand{\itemCourse}{Programación Web 2}
\newcommand{\itemCourseCode}{1702122}
\newcommand{\itemSemester}{III}
\newcommand{\itemUniversity}{Universidad Nacional de San Agustín de Arequipa}
\newcommand{\itemFaculty}{Facultad de Ingeniería de Producción y Servicios}
\newcommand{\itemDepartment}{Departamento Académico de Ingeniería de Sistemas e Informática}
\newcommand{\itemSchool}{Escuela Profesional de Ingeniería de Sistemas}
\newcommand{\itemAcademic}{2024 - A}
\newcommand{\itemInput}{Del 23 de mayo de 2024}
\newcommand{\itemOutput}{Al 15 de julio de 2024}
\newcommand{\itemPracticeNumber}{11}
\newcommand{\itemTheme}{Djangular}
%%%%%%%%%%%%%%%%%%%%%%%%%%%%%%%%%%%%%%%%%%%%%%%%%%%%%%%%%%%%%%%%%%%%%%%%%%%%
%%%%%%%%%%%%%%%%%%%%%%%%%%%%%%%%%%%%%%%%%%%%%%%%%%%%%%%%%%%%%%%%%%%%%%%%%%%%

\usepackage[english,spanish]{babel}
\usepackage[utf8]{inputenc}
\AtBeginDocument{\selectlanguage{spanish}}
\renewcommand{\figurename}{Figura}
\renewcommand{\refname}{Referencias}
\renewcommand{\tablename}{Tabla} %esto no funciona cuando se usa babel
\AtBeginDocument{%
	\renewcommand\tablename{Tabla}
}

\usepackage{fancyhdr}
\pagestyle{fancy}
\fancyhf{}
\setlength{\headheight}{30pt}
\renewcommand{\headrulewidth}{1pt}
\renewcommand{\footrulewidth}{1pt}
\fancyhead[L]{\raisebox{-0.2\height}{\includegraphics[width=3cm]{img/logo_episunsa.png}}}
\fancyhead[C]{\fontsize{7}{7}\selectfont	\itemUniversity \\ \itemFaculty \\ \itemDepartment \\ \itemSchool \\ \textbf{\itemCourse}}
\fancyhead[R]{\raisebox{-0.2\height}{\includegraphics[width=1.2cm]{img/logo_abet}}}
\fancyfoot[L]{Victor M.}
\fancyfoot[C]{\itemCourse}
\fancyfoot[R]{Página \thepage}

% para el codigo fuente
\usepackage{listings}
\usepackage{color, colortbl}
\definecolor{dkgreen}{rgb}{0,0.6,0}
\definecolor{gray}{rgb}{0.5,0.5,0.5}
\definecolor{mauve}{rgb}{0.58,0,0.82}
\definecolor{codebackground}{rgb}{0.95, 0.95, 0.92}
\definecolor{tablebackground}{rgb}{0.8, 0, 0}

\lstset{frame=tb,
	language=bash,
	aboveskip=3mm,
	belowskip=3mm,
	showstringspaces=false,
	columns=flexible,
	basicstyle={\small\ttfamily},
	numbers=none,
	numberstyle=\tiny\color{gray},
	keywordstyle=\color{blue},
	commentstyle=\color{dkgreen},
	stringstyle=\color{mauve},
	breaklines=true,
	breakatwhitespace=true,
	tabsize=3,
	backgroundcolor= \color{codebackground},
}

\begin{document}
	
	\vspace*{10px}
	
	\begin{center}	
		\fontsize{17}{17} \textbf{ Informe de Djangular }
	\end{center}
	\centerline{\textbf{\Large Tema: \itemTheme}}
	%\vspace*{0.5cm}	

	\begin{flushright}
		\begin{tabular}{|M{2.5cm}|N|}
			\hline 
			\rowcolor{tablebackground}
			\color{white} \textbf{Nota}  \\
			\hline 
			     \\[30pt]
			\hline 			
		\end{tabular}
	\end{flushright}	

	\begin{table}[H]
		\begin{tabular}{|x{4.7cm}|x{4.8cm}|x{4.8cm}|}
			\hline 
			\rowcolor{tablebackground}
			\color{white} \textbf{Estudiante} & \color{white}\textbf{Escuela}  & \color{white}\textbf{Asignatura}   \\
			\hline 
			{\itemStudent \par \itemEmail} & \itemSchool & {\itemCourse \par Semestre: \itemSemester \par Código: \itemCourseCode}     \\
			\hline 			
		\end{tabular}
	\end{table}		
	
	\begin{table}[H]
		\begin{tabular}{|x{4.7cm}|x{4.8cm}|x{4.8cm}|}
			\hline 
			\rowcolor{tablebackground}
			\color{white}\textbf{Tarea} & \color{white}\textbf{Tema}  & \color{white}\textbf{Duración}   \\
			\hline 
			\itemPracticeNumber & \itemTheme & 2 horas   \\
			\hline 
		\end{tabular}
	\end{table}
	
	\begin{table}[H]
		\begin{tabular}{|x{4.7cm}|x{4.8cm}|x{4.8cm}|}
			\hline 
			\rowcolor{tablebackground}
			\color{white}\textbf{Semestre académico} & \color{white}\textbf{Fecha de inicio}  & \color{white}\textbf{Fecha de entrega}   \\
			\hline 
			\itemAcademic & \itemInput &  \itemOutput  \\
			\hline 
		\end{tabular}
	\end{table}
%%%%%%%%%%%%%%%%%%%%

  \section{Introducción}
  Djangular combina Django en el backend y Angular en el frontend para crear aplicaciones web modernas 
  y escalables. Django maneja la lógica del servidor y la base de datos, mientras que Angular se encarga 
  de la interfaz de usuario dinámica y interactiva en el navegador. Esta combinación ofrece una estructura 
  poderosa y modular para desarrollar aplicaciones web de alto rendimiento.

%%%%%%%%%%%%%%%%%%%%

  \section{Objetivos}
  \begin{itemize}
    \item Comprender el funcionamiento de Django como BackEnd y Angular como FrontEnd.
    \item Construir aplicaciones escalables separando claramente las responsabilidades entre el backend y el frontend.
    \item Aprovechar las características de seguridad y rendimiento de Django en el servidor para garantizar aplicaciones
    robustas y rápidas.
  \end{itemize}

%%%%%%%%%%%%%%%%%%%%
 
	\section{Tarea}
  \begin{itemize}
    \item Crear un proyecto que combine Angular con Django, repitiendo la experiencia de las clases teóricas y haciendo commit cada avance.  
    \item Compartirlo con el docente (CarloCorrales010).
  \end{itemize}
 
%%%%%%%%%%%%%%%%%%%% 
 
  \section{Entregables}
  \begin{itemize}
    \item Informe hecho en Latex.
    \item URL: Repositorio GitHub.
  \end{itemize}
  
%%%%%%%%%%%%%%%%%%%%    
		
	\section{Equipos, materiales y temas utilizados}
  \begin{itemize}
    \item Django
    \item Angular
    \item RestFrameWork
    \item Api
    \item JSON
  \end{itemize}
 
%%%%%%%%%%%%%%%%%%%%

  \section{URL de Repositorio Github}
  \begin{itemize}
    \item Link de Repositorio GitHub.
    \item \url{https://github.com/Victor-Gonzalo-Maldonado-Vilca/Djangular.git}
  \end{itemize}

%%%%%%%%%%%%%%%%%%%%

  \section{Desarrollo del trabajo}
  \subsection{Capturas de la Actividad Realizada}
  
  
  \subsection{Django}
  \subsubsection{Proyecto djangocrud 'settings.py'}
  \begin{itemize}
    \item \textbf{INSTALLED\_APPS: }Lista las aplicaciones instaladas en Django, incluyendo api, rest\_framework, 
    myapp, y corsheaders. api y myapp son aplicaciones personalizadas, rest\_framework proporciona herramientas 
    para construir APIs web, y corsheaders permite configurar políticas de CORS en la aplicación Django.   
    \begin{lstlisting}[language=python, numbers=left, firstnumber=33, numberstyle=\color{blue}]
    INSTALLED_APPS = [ 
        ...
        'api',
        'rest_framework',
        'myapp',
        'corsheaders',
    ]
    \end{lstlisting}
    \item \textbf{MIDDLEWARE: }Incluye middleware como CorsMiddleware y CommonMiddleware. El CorsMiddleware 
    facilita el manejo de solicitudes de recursos cruzados (CORS), mientras que CommonMiddleware proporciona 
    funcionalidades comunes para el procesamiento de solicitudes HTTP.
    \begin{lstlisting}[language=python, numbers=left, firstnumber=46, numberstyle=\color{blue}]
    MIDDLEWARE = [
        ...
        'corsheaders.middleware.CorsMiddleware',
        'django.middleware.common.CommonMiddleware',
    ]
    \end{lstlisting}
    \item \textbf{CORS\_ALLOWED\_ORIGINS: }Define los orígenes permitidos para las solicitudes CORS, limitando las solicitudes a 
    http://localhost:4200. Esto asegura que la aplicación Django pueda recibir solicitudes AJAX desde el servidor que 
    corre en localhost:4200, típicamente utilizado para desarrollo frontend con frameworks como Angular.
    \begin{lstlisting}[language=python, numbers=left, firstnumber=131, numberstyle=\color{blue}]
    CORS_ALLOWED_ORIGINS = [
        "http://localhost:4200"
    ]
    \end{lstlisting}
  \end{itemize}
  \subsubsection{Proyecto djangocrud 'urls.py'}
  \begin{itemize}
    \item \textbf{Descripción: }Este código configura las rutas esenciales de la aplicación Django con Django REST 
    Framework. Incluye un enrutador (DefaultRouter) para gestionar las URLs de la API, registra MovieViewSet para 
    operaciones CRUD bajo la ruta r'movie', y define rutas para el panel de administración (admin/), la API principal 
    (''), y la autenticación de la API (api-auth/).
    \item \textbf{Código: }
    \begin{lstlisting}[language=python, numbers=left, firstnumber=1, numberstyle=\color{blue}]
    from django.contrib import admin
    from django.urls import path, include
    from rest_framework import routers
    from myapp import views
    router = routers.DefaultRouter()
    router.register(r'movie', views.MovieViewSet)
    urlpatterns = [
        path('admin/', admin.site.urls),
        path('', include(router.urls)),
        path('api-auth/', include('rest_framework.urls', namespace='rest_framework'))
    ]
    \end{lstlisting}
  \end{itemize}
  \subsubsection{Aplicación myapp - api 'models.py'}
  \begin{itemize}
    \item \textbf{Descripción: }El modelo Movie en Django está diseñado para almacenar información básica sobre películas:
    El campo title es de tipo CharField y puede contener hasta 32 caracteres, representando el título de la película.
    El campo desc es también de tipo CharField, con una capacidad máxima de 256 caracteres, utilizado para almacenar una breve descripción de la película.
    El campo year es de tipo IntegerField y se utiliza para almacenar el año de lanzamiento de la película.
    \item \textbf{Código: }
    \begin{lstlisting}[language=python, numbers=left, firstnumber=1, numberstyle=\color{blue}]
    from django.db import models
    # Create your models here.
    class Movie(models.Model):
        title = models.CharField(max_length=32)
        desc = models.CharField(max_length=256)
        year = models.IntegerField()
    \end{lstlisting}
  \end{itemize}
  \subsubsection{Aplicación myapp - api 'admin.py'}
  \begin{itemize}
    \item \textbf{Descripción: }admin.site.register(Movie) en Django registra el modelo Movie en el panel de 
    administración de Django, permitiendo a los administradores gestionar las películas directamente desde 
    una interfaz de usuario administrativa.
    \item \textbf{Código: }
    \begin{lstlisting}[language=python, numbers=left, firstnumber=1, numberstyle=\color{blue}]
    from django.contrib import admin
    from .models import Movie
    # Register your models here.
    admin.site.register(Movie)
    \end{lstlisting}
  \end{itemize}
  \subsubsection{Aplicación myapp - api 'serializer.py'}
  \begin{itemize}
    \item \textbf{Descripción: }El MovieSerializer define cómo los objetos de la clase Movie deben ser serializados 
    y deserializados. Utiliza el modelo Movie como base y especifica los campos id, title, desc, y year que serán 
    incluidos en la representación serializada del objeto. Este serializer es esencial para manejar datos de películas 
    de manera estructurada y eficiente dentro de una API Django, permitiendo la manipulación y visualización coherente 
    de información relacionada con películas a través de solicitudes HTTP.
    \item \textbf{Código: }
    \begin{lstlisting}[language=python, numbers=left, firstnumber=1, numberstyle=\color{blue}]
    from django.contrib.auth.models import User,Group
    from rest_framework import serializers
    from .models import Movie

    class MovieSerializer(serializers.ModelSerializer):
        class Meta:
            model = Movie
            fields = ('id','title','desc','year')
    \end{lstlisting}
  \end{itemize}
  \subsubsection{Aplicación myapp - api 'views.py'}
  \begin{itemize}
    \item \textbf{Descripción: }El MovieViewSet define un conjunto de vistas que manejan las 
    operaciones CRUD para el modelo Movie. Utiliza el queryset Movie.objects.all() para obtener 
    todas las instancias de Movie desde la base de datos. El serializer\_class especifica que el 
    serializador MovieSerializer se utilizará para convertir los objetos Movie en formatos como 
    JSON, permitiendo así que los datos de las películas se manipulen de manera estructurada y 
    consistente a través de solicitudes HTTP dentro de una API Django.
    \item \textbf{Código: }
    \begin{lstlisting}[language=python, numbers=left, firstnumber=1, numberstyle=\color{blue}]
    from django.shortcuts import render
    from django.contrib.auth.models import User,Group
    from rest_framework import viewsets
    from .serializer import MovieSerializer
    from .models import Movie

    class MovieViewSet(viewsets.ModelViewSet):
        queryset = Movie.objects.all()
        serializer_class = MovieSerializer
    \end{lstlisting}
  \end{itemize}
  
  \subsection{Angular}
  \subsubsection{Servicio api.service.ts}
  \begin{itemize}
    \item \textbf{Descripción: }Después de importar los modulos necesarios, tenemos la clase ApiService.Esta clase se 
    encarga de comunicarse con una API RESTful: La propiedad baseurl establece 
    la dirección base de la API en http://127.0.0.1:8000. httpHeaders define los encabezados HTTP necesarios para 
    las solicitudes, con Content-Type configurado como application/json. El constructor de la clase inyecta el 
    servicio HttpClient, permitiendo realizar solicitudes HTTP dentro de los métodos de la clase. El método getAllMovies() 
    utiliza el servicio HttpClient para realizar una solicitud GET a la URL completa de la API concatenada con /movie/, incluyendo los encabezados definidos anteriormente.
    \item \textbf{Código: }
    \begin{lstlisting}[language=java, numbers=left, firstnumber=1, numberstyle=\color{black}]
    import { Injectable } from '@angular/core';
    import { HttpClient, HttpHeaders } from '@angular/common/http';
    import { Observable } from 'rxjs';

    @Injectable({
      providedIn: 'root'
    })
    export class ApiService {

      baseurl = "http://127.0.0.1:8000";
      httpHeaders = new HttpHeaders({'Content-Type':'application/json'});
      constructor(private http:HttpClient) { }
      getAllMovies():Observable<any>{
        return this.http.get(this.baseurl+'/movie/', 
        {headers: this.httpHeaders});
      }
    }
    \end{lstlisting}
  \end{itemize}
  \subsubsection{Componente app.component.ts}
  \begin{itemize}
    \item \textbf{Descripción: }Luego de hacer las importaciones correspondientes se define la clase AppComponent. 
    La propiedad movies es un arreglo de objetos de películas inicializado localmente para demostración. 
    El constructor inyecta el servicio ApiService y llama al método getMovies() al inicializar el componente. 
    El método getMovies() utiliza el servicio ApiService para obtener películas desde la API. Los datos devueltos 
    por getAllMovies() se registran en la consola.
    \item \textbf{Código: }
    \begin{lstlisting}[language=java, numbers=left, firstnumber=10, numberstyle=\color{black}]
    export class AppComponent {
      movies = [{id:1,title:'peli1',desc:"50%",year:2021},{id:2,title:'peli2', desc:'50%',year:2022}];
      constructor(private api:ApiService) {
        this.getMovies();
      }
      getMovies = () => {
        this.api.getAllMovies().subscribe (
          data => {
            console.log(data);
            this.movies = data;  //data.results;
          },
          error => {
            console.log(error);
          }) 
      } 
    }
    \end{lstlisting}
  \end{itemize}
  \subsubsection{html app.component.html}
  \begin{itemize}
    \item \textbf{Descripción: }Se muestra una lista de películas. Utiliza <router-outlet> como un marcador de posición para renderizar 
    componentes según las rutas definidas en la aplicación Angular. A continuación, un encabezado \verb|(<h2>)| dice 
    \verb|"Lista de Movies:"|. Luego, se muestra una lista desordenada \verb|(<ul>)| que itera sobre el array movies utilizando 
    *ngFor. Para cada película en el array, se crea un elemento de lista \verb|(<li>)| que muestra el ID, título, descripción 
    y año de la película dentro de un encabezado de segundo nivel \verb|(<h2>)|.
    \item \textbf{Código: }
    \begin{lstlisting}[language=html, numbers=left, firstnumber=1, numberstyle=\color{orange}]
    <router-outlet></router-outlet>
    <!--Mi lista aqui va -->
    <h2>Lista de Movies:</h2>
    <ul>
      <li *ngFor="let movie of movies">
        <h2> {{movie.id }}: Titulo: {{ movie.title }} <br>
          Descripcion: {{ movie.desc }}. year: {{ movie.year }}</h2>
      </li>
    </ul>
    \end{lstlisting}
  \end{itemize}

%%%%%%%%%%%%%%%%%%%%

  \section{Conclusiones}
  \begin{itemize}
    \item La integración de Django y Angular permite una aplicación web robusta que utiliza Django como backend 
    y Angular como frontend.
    \item Django proporciona una API RESTful que Angular consume para mostrar datos dinámicamente.
    \item El uso de \texttt{router-outlet} en Angular facilita la navegación y el enrutamiento dentro de la aplicación.
    \item Las listas dinámicas en Angular, como la lista de películas, demuestran la capacidad de Angular para manejar 
    y mostrar datos obtenidos de la API de Django.
    \item Esta combinación de tecnologías permite una separación clara entre el backend y el frontend, facilitando el 
    desarrollo y mantenimiento de la aplicación.
  \end{itemize}


%%%%%%%%%%%%%%%%%%%%
	\newpage
	\subsection{\textcolor{red}{Rúbrica para el contenido del Informe y demostración}}
	\begin{itemize}			
		\item El alumno debe marcar o dejar en blanco en celdas de la columna \textbf{Checklist} si cumplio con el ítem correspondiente.
		\item Si un alumno supera la fecha de entrega,  su calificación será sobre la nota mínima aprobada, siempre y cuando cumpla con todos lo items.
		\item El alumno debe autocalificarse en la columna \textbf{Estudiante} de acuerdo a la siguiente tabla:
	
		\begin{table}[ht]
			\caption{Niveles de desempeño}
			\begin{center}
			\begin{tabular}{ccccc}
    			\hline
    			 & \multicolumn{4}{c}{Nivel}\\
    			\cline{1-5}
    			\textbf{Puntos} & Insatisfactorio 25\%& En Proceso 50\% & Satisfactorio 75\% & Sobresaliente 100\%\\
    			\textbf{2.0}&0.5&1.0&1.5&2.0\\
    			\textbf{4.0}&1.0&2.0&3.0&4.0\\
    		\hline
			\end{tabular}
		\end{center}
	\end{table}	
	

	\end{itemize}

 
	
	\begin{table}[H]
		\caption{Rúbrica para contenido del Informe y demostración}
		\setlength{\tabcolsep}{0.5em} % for the horizontal padding
		{\renewcommand{\arraystretch}{1.5}% for the vertical padding
		%\begin{center}
		\begin{tabular}{|p{2.7cm}|p{7cm}|x{1.3cm}|p{1.2cm}|p{1.5cm}|p{1.1cm}|}
			\hline
    		\multicolumn{2}{|c|}{Contenido y demostración} & Puntos & Checklist & Estudiante & Profesor\\
			\hline
			\textbf{1. GitHub} & Hay enlace URL activo del directorio para el  laboratorio hacia su repositorio GitHub con código fuente terminado y fácil de revisar. &2 &X &2 & \\ 
			\hline
			\textbf{2. Commits} &  Hay capturas de pantalla de los commits más importantes con sus explicaciones detalladas. (El profesor puede preguntar para refrendar calificación). &4 &X &4 & \\ 
			\hline 
			\textbf{3. Código fuente} &  Hay porciones de código fuente importantes con numeración y explicaciones detalladas de sus funciones. &2 &X &2 & \\ 
			\hline 
			\textbf{4. Ejecución} & Se incluyen ejecuciones/pruebas del código fuente  explicadas gradualmente. &2 &X &2 & \\ 
			\hline			
			\textbf{5. Pregunta} & Se responde con completitud a la pregunta formulada en la tarea.  (El profesor puede preguntar para refrendar calificación).  &2 &X &2 & \\ 
			\hline	
			\textbf{6. Fechas} & Las fechas de modificación del código fuente estan dentro de los plazos de fecha de entrega establecidos. &2 &X &2 & \\ 
			\hline 
			\textbf{7. Ortografía} & El documento no muestra errores ortográficos. &2 &X &2 & \\ 
			\hline 
			\textbf{8. Madurez} & El Informe muestra de manera general una evolución de la madurez del código fuente,  explicaciones puntuales pero precisas y un acabado impecable.   (El profesor puede preguntar para refrendar calificación).  &4 &X &4 & \\ 
			\hline
			\multicolumn{2}{|c|}{\textbf{Total}} &20 & &20 & \\ 
			\hline
		\end{tabular}
		%\end{center}
		%\label{tab:multicol}
		}
	\end{table}


%%%%%%%%%%%%%%%%%%%%%%%%%%%%%%%%%%%%%%%%%%%%%%%%%%%%%%%%%%%%%%%%%%%
	
  \newpage
  \section{Referencias}
  \begin{itemize}
    \item \url{https://www.djangoproject.com/}
    \item \url{https://v17.angular.io/guide/architecture}
  \end{itemize}

%%%%%%%%%%%%%%%%%%%% 
%\clearpage
%\bibliographystyle{apalike}
%\bibliographystyle{IEEEtranN}
%\bibliography{bibliography}
			
\end{document}
